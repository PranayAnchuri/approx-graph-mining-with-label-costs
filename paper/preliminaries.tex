\section{Preliminaries}

An undirected labeled graph \db is represented as a tuple $ \db =
(\vg,\eg,L) $ where $\vg$ is the set of vertices, $\eg$ is the set of
edges and $L\!\!: \vg \rightarrow \Sigma $ is a function that maps
vertices to their labels.  The neighbors of a vertex $v$ are given as $
N(v) = \{ u | (u,v) \in \vg \} $, and its degree as $d(v) = |N(v)|$.  A
{\em walk} in a graph $\db$ is a sequence of vertices $v_0,\ldots,v_k$
such that $v_i \in \vg$ and $(v_i,v_{i+1}) \in \eg$. It covers the
vertices $\cup_{i=0}^{k} \{v_i\}$ and the edges $\cup_{i=0}^{k-1}
\{(v_i,v_{i+1})\}$.  A path is a walk without any repeated vertices,
i.e., $v_i \ne v_j$ for $i \ne j$,and we say that $v_{k}$ is 
{\em reachable} from $v_0$ in \khops.

\smallskip\noindent{\bf Cost matrix:}
We assume that there exists a cost function $cost: \Sigma \times \Sigma
\to \setR$, where $cost(l_i,l_j)$ is the cost of matching the labels
$l_i$ and $l_j$. The cost matrix \M{C} is a $|\Sigma| \times |\Sigma|$
matrix where $\matij{C}{l_i}{l_j} = cost(l_i,l_j)$. For simplicity, we
assume that $\matij{C}{l_i}{l_i} = 0 $ and $\matij{C}{l_i}{l_j} > 0$.
The cost matrix is not required to be symmetric. 
% Is it required to mention the binary version here ?
In some application domains it is difficult to construct a meaningful cost
matrix.  However, it may be easier to specify which labels are very
similar to other labels, which leads to a binary cost matrix, where 
$\matij{C}{l_i}{l_j} = 0$ iff $l_i$ and $l_j$ are highly similar, and 
$\matij{C}{l_i}{l_j} = 1$ otherwise (alternatively, the cost may be
$\infty$ if $l_i$ and $l_j$ are not similar). 

\smallskip\noindent{\bf Approximate subgraph isomorphism:}
A graph $S = (V_S,E_S,L)$ is a subgraph of \db, denoted $S \subseteq
\db$, iff $V_S \subseteq \vg$, and $E_S \subseteq \eg$.  $S$ is an
induced subgraph if $E_S = \eg \cap (V_S \times V_S)$.  Given a database
graph $G$ and a pattern $P = (V_P,E_P,L)$, a function $\phi\!\!: V_P \to
V_G$ is called an {\em unlabeled subgraph isomorphism} provided $\phi$
is an injective (or one-to-one) mapping such that $\forall (u,v) \in
E_P$, we have $(\phi(u),\phi(v)) \in \eg$. That is, $\phi$ preserves the
topology of $P$ in $G$. Define the cost of the isomorphism as follows:
$C(\phi) = \sum_{u \in \vp} \matij{C}{L(u)}{L(\phi(u))}$, that is, the
sum of the costs of matching the node labels in $P$ to the corresponding
node labels in $G$.  We say that $\phi$ is an {\em approximate subgraph
isomorphism} from $P$ to $G$ provided its cost $C(\phi) \le \alpha$,
where $\alpha$ is a user-specified threshold on the total cost. In this
case we also call $P$ an approximate pattern in $G$. Note
that if $\alpha = 0$, then $\phi$ is an exact subgraph isomorphism
between $P$ and $G$.


%%% Example graphs to illustrate the notation used in the paper
\begin{figure}[!h]
\vspace{-0.25in}
\captionsetup[subfloat]{captionskip=15pt}
  \subfloat[Database Graph $\db$] {
    \label{subfig:ex_db}
  \scalebox{0.9}{
    \begin{pspicture}(-1,0)(3,3)
      \putNode{1}{2}{n1}{A}{90}{1}
      \putNode{0}{1}{n2}{A}{180}{2}
      \putNode{1}{1}{n3}{B}{225}{3}
      \putNode{2}{1}{n4}{A}{0}{4}
      \putNode{0}{0}{n5}{C}{270}{5}
      \putNode{1.5}{0}{n6}{C}{270}{6}

      \ncline{-}{n1}{n2}
      \ncline{-}{n1}{n3}
      \ncline{-}{n1}{n4}
      \ncline{-}{n2}{n3}
      \ncline{-}{n3}{n4}
      \ncline{-}{n3}{n6}
      \ncline{-}{n4}{n6}
      \ncline{-}{n2}{n5}
    \end{pspicture}
	}}
  % Graph subgraph isomorphic to the database
  \subfloat[Pattern $P$] {
    \label{subfig:ex_sub}
  \scalebox{0.9}{
    \begin{pspicture}(0.5,1)(3,3)
      \putNode{2}{3}{n1}{A}{90}{10}
      \putNode{1}{2}{n2}{B}{180}{20}
      \putNode{3}{2}{n3}{C}{0}{30}
      \putNode{2}{1}{n4}{A}{0}{40}
      \ncline{-}{n1}{n2}
      \ncline{-}{n1}{n3}
      \ncline{-}{n2}{n4}
      \ncline{-}{n3}{n4}
    \end{pspicture}
	}}
  \newline
\captionsetup[subfloat]{captionskip=5pt}
  \subfloat[Cost Matrix] {
    \label{subfig:ex_match}
    \begin{tabular}{|c|c|c|c|c|}
      \hline
      \M{C} & A &  B & C & D \\
      \hline
      A & $0$ & $0.2$ & $0.6$ & $0.1$ \\
      \hline
      B & $0.2$ & $0$ & $0.4$ & $0.5$ \\
      \hline
      C & $0.6$ & $0.4$ & $0$ & $0.2$ \\
      \hline
      D & $0.1$ & $0.5$ & $0.2$ & $0$ \\
      \hline
    \end{tabular}
  }
  % Approximately isomorphic
  \subfloat[Approximate Embeddings] {
      \label{subfig:ex_occur}
      \begin{tabular}{|l|c|c|c|c|c|}
		\hline
		 & \multicolumn{5}{|c|}{$\phi$}\\ 
        \hline
		$P$ & $10$ & $20$ & $30$ & $40$ & cost \\
        \hline
        $\phi_1$ & $3$ & $1$ & $6$ & $4$ & $0.4$ \\
        $\phi_2$ & $4$ & $1$ & $6$ & $3$ & $0.4$\\
        \hline
      \end{tabular}
  }
  % occurrences of the approx pattern
  \caption{ \protect\subref{subfig:ex_db}: sample database graph $G$, 
    \protect\subref{subfig:ex_sub}: approximate pattern $P$.
	\protect\subref{subfig:ex_match}: cost matrix.
	\protect\subref{subfig:ex_occur}: approximate embeddings 
	of $P$ in $G$.
  }
  \label{fig:ex1}
\end{figure}

\smallskip\noindent{\bf Representative set and pattern support:}
Given a node $u \in \vp$, its {\em representative set} in the database
graph $G$ is the set 
$$R(u) = \{ v \in \vg |\; \exists \phi, \text{ such
that } C(\phi) \le \alpha \text{ and } \phi(u) = v \}$$ 
That is, the representative set of $u$ comprises all nodes in $G$ that
$u$ is mapped to in some approximate isomorphism.  
Figure
\ref{fig:ex1} shows an example database, a cost matrix, an approximate
pattern, and its approximate subgraph isomorphism for $\alpha=0.5$.
There are only two possible approximate isomorphisms from $P$ to $G$, as
specified by $\phi_1$ and $\phi_2$. For example, for $\phi_1$, we have
$\phi_1(10) \to 3$, $\phi_1(20) \to 1$, $\phi_1(30) \to 6$, and
$\phi_1(40) \to 4$, as seen in Table~\ref{subfig:ex_occur}. 
The cost of the isomorphism is 
$C(\phi_1) = 0.4$, since 
$C(L(10),L(3)) + C(L(20),L(1)) + C(L(30),L(6)) + C(L(40),L(4)) 
= C(A,B) + C(B,A) + C(C,C)+ C(A,A) = 0.2+0.2+0+0 = 0.4$. 
The representative
set for node $10 \in \vp$ is $R(10) = \{3, 4\}$. However, the
support of $P$ is $sup(P) = 1$, since node $20 \in \vp$ has only one
mapping in $G$, namely $R(20) = \{1\}$.


%%%%%%%%% the definition of support is moved to a later section %%%%%%
\if0
Define the {\em
support} of pattern $P$ in a database graph $G$ as 
$$sup(P) = \min_{u \in
\vp} \{ |R(u)| \}$$
That is, the minimum cardinality over all
representative set of nodes in $P$.  A pattern $P$ is called {\em
frequent} if $sup(P) \geq minsup$, where $minsup$ is a user defined
support threshold.  $P$ is {\em maximal} iff $P$ is frequent and there
does not exist any supergraph of $P$ that is frequent in $G$.  

\smallskip
\noindent{\bf Problem Statement:} Given a database graph \db, minimum
support $minsup$, maximum allowed cost $\alpha$, and an integer $k$, our
goal is to mine $k$ maximal frequent approximate patterns.  Figure
\ref{fig:ex1} shows an example database, a cost matrix, an approximate
pattern, and its approximate subgraph isomorphism for $\alpha=0.5$.
There are only two possible approximate isomorphisms from $P$ to $G$, as
specified by $\phi_1$ and $\phi_2$. For example, for $\phi_1$, we have
$\phi_1(10) \to 3$, $\phi_1(20) \to 1$, $\phi_1(30) \to 6$, and
$\phi_1(40) \to 4$, as seen in Table~\ref{subfig:ex_occur}. 
The cost of the isomorphism is 
$C(\phi_1) = 0.4$, since 
$C(L(10),L(3)) + C(L(20),L(1)) + C(L(30),L(6)) + C(L(40),L(4)) 
= C(A,B) + C(B,A) + C(C,C)+ C(A,A) = 0.2+0.2+0+0 = 0.4$. 
The representative
set for node $10 \in \vp$ is $R(10) = \{3, 4\}$. However, the
support of $P$ is $sup(P) = 1$, since node $20 \in \vp$ has only one
mapping in $G$, namely $R(20) = \{1\}$.
\fi
