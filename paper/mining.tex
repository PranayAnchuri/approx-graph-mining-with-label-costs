% describe the other modules required for the mining process
\section{Mining Algorithm}
\label{sec:mining}
 The mining algorithm involves candidate generation, support computation in
 addition to finding the set of isomorphisms.  Representative sets of vertices
 described in the previous section are a compact view of all the isomorphisms of
 the pattern in the input graph. In this section, we will show how the
 representative sets can be used in conjunction with different candidate
 generation and support computation techniques to yield approximate graph mining
 algorithm with different properties.


\subsection{Candidate Generation} 

The search space of the frequent patterns forms a partial order.  It can be
explored in a depth first or breadth first order but doing so requires computing
canonical code to avoid duplicates. Since, the search space is exponential,
sampling methods have gained traction in recent
times~\cite{2008-origami:sadm,2009-graphsampling} . 


In our experiments we employed the random walk strategy proposed in
\cite{2011-icdm} to mine exact patterns from a single large graph. Each random
walk starts with an empty pattern and repeatedly adds new edges (to new
vertices) or connects two existing vertices in the pattern to generate a new
candidate. At any stage of the walk let $Q$ be the current frequent pattern.
A candidate pattern $P$ is generated from $Q$ either by adding a new vertex 
with label $l$ or by connecting two existing vertices $u, v \in V_Q$.
For any vertex $u$, if $u \in V_P \cap V_Q$ then the candidate representative set 
$R'(u)$ in $P$ is same as representative set $R(u)$ verified for $Q$. Otherwise
$u \in V_P \setminus V_Q$ and the candidate representative set is $R'(u) = \{ v
| v \in \vg, \matij{C}{L(u)}{L(v)} \leq \alpha \} $ i.e., we start with the
current representatives if the vertex is already present otherwise it is the set
of vertices in $\db$ whose label matching cost is within $\alpha$. Using the
label pruning and verification mechanism we compute the representatives of $P$.
Then we decide if the pattern is frequent using the support function that we
will define in section~\ref{sec:support}. If the candidate pattern $P$ is
frequent, then we continute the walk by extending $P$. Otherwise, we  try
another extension from $Q$. If no extension of $Q$ leads to a frequent pattern
then $Q$ is a maximal and we terminate the current random walk.
The algorithm terminates when $K$ walks have been done, or when $K$
distinct maximal approximate patterns have been output. But, if the application
requires a complete set of maximal patterns an ordered exploration of the search
space may be employed.


\subsection{Support Computation} \label{sec:support} The support of a pattern is
an anti-monotonic function on the set of isomorphisms of the pattern. The
anti-monotonicity means that the support of a pattern cannot be greater than the
support of any of its subgraph. Therefore, if a candidate pattern is found to be
infrequent we can prune the entire sub tree under it from the search space.
This helps in pruning the otherwise exponential search space. 

When mining from a database of graphs, a function as simple as the total number
of graphs having at least one isomorphism is anti-monotonic. This approach cannot
be used when mining from a single graph as it leads to a binary support function
which is not very informative. On the other hand, counting the number of
isomorphisms is not anti-monotonic because a graph can have more isomorphisms
compared to its sub graph.

An anti-monotonic support function for a single graph is the maximum number of
vertex disjoint isomorphisms. However, this requires computing the maximum
independent set (MIS) of graph where each vertex represents an isomorphism.
This is called the MIS support of the pattern.
Clearly, it is not feasible when the input graphs are large and patterns have
large number of isomorphisms. An easy upper bound on the MIS support is the size
of the smallest representative set of a vertex in the pattern.  Define the {\em
support} of pattern $P$ in a database graph $G$ as $$sup(P) = \min_{u \in \vp}
\{ |R(u)| \}$$ That is, the minimum cardinality over all representative sets of
vertices in $P$.  The size of representative sets constructed from the disjoint
isomorphisms is equal to the MIS support. Hence, $sup(P)$ is at least as large
MIS support.  Other upper bounds for the MIS value have been proposed in gApprox
and CMDB-Miner algorithms. The support function used in gApprox can be computed
from the representative sets by enumerating the isomorphisms as described in the
Section~\ref{sec:verification}.  The support function used by the CMDB-Miner
algorithm can also be used by constructing an appropriate flow network on the
representative sets.

In conclusion, we can mix and match different techniques for candidate
generation and support computation to produce different versions of the
approximate graph mining algorithm even though the isomorphisms are stored as
representatives.

\subsection{Complexity}
\medskip{\textit{Space Complexity} :} At given stage of the mining process,
we need to store the candidate representative sets and the precomputed 
\khop labels. For a pattern with $m$ vertices, the total amount of memory
is O($m \times |\vg| + k_{max} \times ( |\Sigma| * |\vg|)$). The first term
correponds to the representative sets and the second to the precomputed
\khop labels. $k_{max}$ is the maximum value of $k$ for which we compute \khop
labels.

\medskip{\textit{Time Complexity} : } The cost of matching the \khop labels
requires $|\khopl{k}{u}|$ augmentations in the flow network $F$ which is an
upper bound on the min cost assuming the cost of each edge is at most one. Each
augmentation involves cycle detection which takes O($|\Sigma|^3$) because the
number of vertices in $F$ is O($|\Sigma|$).  The time for the bipartite matching
in a graph is proportional to the number of vertices and the edges in it.
Since, we try to match the neighbors of pattern and candidate vertex the number
of vertices is bounded by the maximum degree $d_{max}$ of the pattern and
candidate vertices. Therefore, the total time for each dominance check is
O($|\khopl{k}{u}| \times |\Sigma|^3 + d_{max}^{3}$).  The number of dominance
checks performed to per candidate are O($k_{max} \times n_g \times |\vg|$) where
$n_g$ is the number of orbit groups in the pattern vertex. 

%%%%%%%%%%%%%%%%%%%%%%%% Comment%%%%%%%%%%%%%%%%%%%%%%%%%%
\if0
To compute the support of an approximate pattern $P$ in a database \db,
existing algorithms (e.g., gApprox~\cite{gapprox}) maintain the list of
all approximate isomorphisms of $P$, and then for each extension they
check which of the existing isomorphisms can be extended. However, such an
approach is inherently inefficient, since the number of isomorphisms can
be exponentially large~\cite{2011-icdm}.  Thus, any approach that
enumerates all the approximate isomorphisms scales very poorly as the size
of the database graph increases, and is also highly dependent on the
multiplicity of the labels (i.e., the number of neighbors with the same
label) for nodes in the database.

Our approach is entirely different. Instead of storing all isomorphisms
for a pattern, we keep track of only the representative set for each
node in the pattern.  Since the support of a pattern is bounded above by
$N=|\vg|$ (which follows immediately from the fact that $R(u) \subseteq
\vg$ for any $u \in \vp$), the total size of the representative set of a
candidate cannot exceed $|\vp| \times N$ (practically, it will be much
smaller).  Let $P'$ be an extension of a pattern $P$ obtained by adding
one more edge. Let $R(u)$ be the representative set for a node $u$ in
$P$, and let $R'(u)$ be the representative set for the same node in the
extension $P'$. It is clear that $R'(u) \subseteq R(u) \subseteq \vg$.
So our method tries to prune vertices from the new candidate
representative set $R'(u)$. 
Furthermore, we can also obtain an upper
bound on the support of $P'$, since $sup(P') \le \min_{u \in V_{P'}}
\{|R'(u)|\}$. Thus, if the upper bound itself is not frequent, we can
also prune the pattern search space.
\fi
