%s How the representative set is computed; pruning strategies and verification
%step goes here.

\section{Computing Representative Sets} Representative vertex $v$ of a pattern
vertex $u$ implies that there exists an isomorphism $\phi$ for which $\phi(u) =
v$.  One way to interpret it is that the neighborhood of $u$ matches with that
of $v$.  By comparing the neighborhoods we can find vertices that are not valid
representatives of $u$ without trying to find an isomorphism exhaustively.
Therefore, to compute the representative sets we will start with a candidate
representative set denoted by \CR  and iteratively prune some of the vertices if
the neighborhoods cannot be matched.  The candidate set is a superset of the
representative set, $\CR \supseteq \RS$.  An example of candidate set, $ \CR =
\{v| v \in \vg, \matij{C}{L(u)}{L(v)} \leq \alpha \}$ i.e., the isomorphisms of
the single vertex pattern with label $L(u)$.  In this section, we will describe
different notions of neighborhood and show how they help us in computing the
representative sets of vertices in a pattern.

The problem of checking whether a vertex $v \in R(u)$ involves solving
isomorphism which is a NP complete problem. The pruning methods typically do not
prune all the invalid vertices.  So, we use an exhaustive enumeration method to
prune these invalid vertices and reduce \CR to \RS.

 \subsection{\khop Label}
 %Neighbors of a vertex in a graph denotes the set of vertices that are
 %reachable via a single edge. 
 \khop label is defined as the set of vertices that are reachable via a simple
 path of length $k$.  In other words, k-hop label contains all vertices that are
 reachable in k-hops starting from $u$ and by visiting each vertex at most once.
 Note that, we use the word label even though we refer to a set of vertices.
 Formally, the \khop label of a vertex $u$ in graph $G$, $\khopl{k}{u,G} = \{v |
 v \in G, \kpath{u}{v}{k}\}$.  We simply write it as $\khopl{k}{u}$ when the
 graph is evident from the context.  For example, for pattern $P$ in
 Fig.~\ref{subfig:pattern}, the $0$-hop label of vertex $5$ is $h_0(5) = \{5\}$,
 its $1$-hop label is the multiset $h_1(5) = 2, 4, 6$ (we omit the set notation
 for convenience) and its $2$-hop label $h_2(5) = 1, 3$. The minimum cost of
 matching \khop labels $\khopl{k}{u}$ and $\khopl{k}{v}$ is 

 \begin{equation} \label{eq:khop} \khopcost{k}{u}{v} =
     \text{min}\displaystyle\sum_{u' \in \khopl{k}{u}}
     \matij{C}{L(u')}{L(f(u'))} \end{equation}

 where the minimization is over all injective functions $f\!\!:\khopl{k}{u}
 \rightarrow \khopl{k}{v}$ and $\matij{C}{L(u')}{L(f(u'))}$ is the cost of
 matching the vertex labels.  In other words, it is the minimum total cost of
 matching the vertices present in the k-hop labels.  The following theorem
 places a upper bound on the minimum cost of matching the k-hop labels of
 pattern vertex and any of its representative vertices.


\begin{thm} Given any pattern vertex $u$, a representative vertex $v \in R(u)$
    and cost threshold $\alpha$, the minimum cost of matching the \khop labels,
    $\khopcost{k}{u}{v} \leq \alpha$ for all $k \geq 0$.

\begin{myproof} Consider any isomorphism $\phi$ such that $\phi(u) = v$. It is
    enough if we can show an injective function $f\!\!:\khopl{k}{u} \rightarrow
    \khopl{k}{v}$ with a cost ( as defined in equation \ref{eq:khop}) $\leq
    \alpha$. We will argue that the function $\phi$ on the restricted domain
    $\khopl{k}{u}$ is one such function $f$.  First, we know that
    $\sum\matij{C}{L(u)}{\phi(L(u)} \leq \alpha$, $u \in \vp$, since $\phi$ is
    an isomorphism. Second, let $\kpath{u}{u'}{k}$ then $\phi(u') \in
    \khopl{k}{v}$ because for every edge $(u_1, u_2)$ on a path between $u$ and
    $u'$ in $\pat$, $(\phi(u_1),\phi(u_2)) \in \eg$.  Therefore the minimum cost
    of matching the \khop labels is upper bounded by $\alpha$.  \end{myproof}
    \label{thm:khop} \end{thm}

Based on the above theorem, a vertex $v$ is an invalid vertex if
$\khopcost{k}{u}{v} > \alpha$ for any $k \geq 0$. However, in practice, it
enough to check the condition only for $k \leq |V_P|-1$ because $\khopl{k}{u}$
is the null set $\forall k \geq |V_P|$ and the condition is trivially satisfied.

Figure~\ref{fig:ncexample} shows an example for the \khop label based pruning of
the candidate representative set where the threshold $\alpha = 0.5$. Consider
vertex $2 \in \vp$ and vertex $20 \in \vg$, we have, $\khopcost{0}{2}{20} = 0$,
since the cost of matching vertex labels $\matij{C}{L(2)}{L(20)} = 0$ , as per
the label matching matrix $C$ in Fig.~\ref{subfig:match}. The \khop labels for
$k=1,2,3$ and the minimum of cost matching them are as shown in the table
\ref{tab:khop220}, and it can be verified that the minimum cost is within the
threshold $\alpha$.

Thus far, we cannot prune node $20$ from $R'(2)$.  However, $\khopl{4}{2} = 4, 5
$ and $\khopl{4}{20} = 30, 60$ and the minimum cost of matching them is $0.6 >
\alpha$.  Thus, we conclude that $20 \notin R'(2)$. This example illustrates
that \khop labels can help prune the candidate representative sets.


% Example for showing the incremental updates of the labels
\begin{figure}[!ht]
\captionsetup[subfloat]{captionskip=15pt}
  \centering
  \subfloat[Pattern $P$]{
    \label{subfig:pattern}
	\scalebox{0.9}{
    % # 3 vertices and 2 edge pattern
    \begin{pspicture}(0,0)(4,3)
    \cnodeput[linecolor=black](0,2) {n1} {A}
    \cnodeput[linecolor=black](0,1) {n2} {C}
    \cnodeput[linecolor=black](1,0) {n3} {B}
    \cnodeput[linecolor=black](2,2) {n4} {C}
    \cnodeput[linecolor=black](2,1) {n5} {A}
    \cnodeput[linecolor=black](2,0) {n6} {D}
    %% Draw the edges of the pattern
  \ncline{-}{n1}{n2}
  \ncline{-}{n2}{n3}
  \ncline{-}{n3}{n4}
  \ncline{-}{n2}{n5}
  \ncline{-}{n4}{n5}
  \ncline{-}{n5}{n6}
  \ncline{-}{n3}{n6}
    \uput{.3cm}[90](n1){ {1} }
    \uput{.3cm}[180](n2){ {2} }
    \uput{.3cm}[270](n3){ {3} }
    \uput{.3cm}[90](n4){ {4} }
    \uput{.3cm}[0](n5){ {5} }
    \uput{.3cm}[270](n6){ {6} }
    \end{pspicture}
	} }
  \subfloat[Database Graph $G$]{
    \label{subfig:database}
	\scalebox{0.9}{
    \begin{pspicture}(0,0)(2.5,2.5)
    \cnodeput[linecolor=black](0,2) {N1} {A}
    \cnodeput[linecolor=black](0,1) {N2} {C}
    \cnodeput[linecolor=black](0,0) {N3} {D}
    \cnodeput[linecolor=black](1,2) {N4} {B}
    \cnodeput[linecolor=black](2.25,1) {N5} {B}
    \cnodeput[linecolor=black](2,0) {N6} {A}
    % vertex ids
    \uput{.3cm}[90](N1){ {10} }
    \uput{.3cm}[180](N2){ {20} }
    \uput{.3cm}[270](N3){ {30} }
    \uput{.3cm}[90](N4){ {40} }
    \uput{.3cm}[0](N5){ {50} }
    \uput{.3cm}[270](N6){ {60} }
    % edges in the database
  \ncline{-}{N1}{N2}
  \ncline{-}{N2}{N3}
  \ncline{-}{N4}{N5}
  \ncline{-}{N5}{N6}
  \ncline{-}{N3}{N4}
  \ncline{-}{N2}{N5}
  \ncline{-}{N2}{N6}
    \end{pspicture}
	} }
  \newline
\captionsetup[subfloat]{captionskip=5pt}
\subfloat[Cost Matrix]{
  \label{subfig:match}
  % Table for the search space pruning
  \begin{tabular}{|c|c|c|c|c|}
    \hline
    $C$ & A &  B & C & D \\
    \hline
    A & 0 & 0.7 & 0.6 & 0.1\\
    \hline
    B & 0.7 & 0 & 0.3 & 1\\
    \hline
    C & 0.6 & 0.3 & 0 & 0.8\\
    \hline
    D & 0.1 & 1 & 0.8 & 0\\
    \hline
  \end{tabular}
  } 
  \caption{Pattern \protect\subref{subfig:pattern}, 
  database graph \protect\subref{subfig:database}, and binary cost
  matrix \protect\subref{subfig:match}.
  } 
  \label{fig:ncexample}
\end{figure}

\begin{table}[h]
    \centering
    \begin{tabular}{|c|c|c|c|}
        \hline
        k & $\khopl{k}{2}$ & $\khopl{k}{20}$ & $\khopcost{k}{2}{20}$\\
        \hline
        1 & 1, 3, 5 & 10, 30, 50, 60 & 0 \\
        2 & 4, 6 & 40, 50, 60 & 0.4 \\
        3 & 3, 5 & 40, 30, 50 & 0.1\\
        \hline
    \end{tabular}
    \caption{\khop label of vertices $2$ and $20$}
    \label{tab:khop220}
\end{table}

\begin{table}[h]
    \centering
    \begin{tabular}{|c|c|c|c|}
        \hline
        k & $h_k(3)$ & $h_k(50)$ & $\khopcost{k}{3}{50}$ \\
        \hline
        0 & 3 & 50 & $0$\\
        1 & 2, 4, 6 & 20, 40, 60 & $0.4$ \\
        2 & 1, 5 & 10, 20 , 30, 60 & 0\\
        3 & 2, 4, 5 & 10, 20, 30, 40 & $0.3$ \\
        4 & $1$ & $10, 40, 60$ & $0$ \\
        \hline
    \end{tabular}
    \caption{\khop labels of vertices $3$ and $50$.}
    \label{tab:khop350}
\end{table}



\subsection{Neighbor Concatenated Label} In Neighbor concatenated label (\ncl) ,
the information regarding the candidates of a neighbor that were pruned in the
previous iteration is used along with the current \khop label to prune
candidates in the current iteration. In contrast, the \khop label pruning
strategy for a vertex $u$ works independently of the result of \khop label
pruning of other vertices in the pattern. This leads us to the following
recursive formulation for \ncl.

The \ncl of a vertex in the ${k+1}^{th}$ iteration, $\nclab{k+1}{u}$, is defined
as the tuple $(\{\nclab{k}{u'} | u' \in N(u)\},\xspace \khopl{k+1}{u})$.  The
first element(A) of the tuple is the \ncl of the neighbors of the vertex $u$ in
the previous iteration and the second element(B) is exactly same as the \khop
label defined in the previous section. We say that $\nclab{k+1}{v}$ dominates
$\nclab{k+1}{u}$, denoted by $\nclab{k+1}{u} = (A, B) \preceq \nclab{k+1}{v} =
(A', B') $, i) iff $\khopcost{k+1}{B}{B'} \leq \alpha$ i.e., the minimum cost of
matching the \khop labels is within $\alpha$ ii) there exits an injective
function $g\!\!:A\rightarrow A'$ such that $a \preceq g(a)$ for all $a \in A$
i.e., there is a one to one mapping between the \ncl labels (in the previous
iteration, $k$) of neighbors of $u$ and $v$.  The base case $\nclab{0}{u}
\preceq \nclab{0}{v}$ iff $\matij{C}{L(u)}{L(v)} \leq \alpha$. For example, in
Fig~\ref{fig:ncexample} $\nclab{1}{2} \preceq \nclab{1}{20}$ because
$\khopcost{1}{2}{20} \leq \alpha$ and the \ncl labels of vertices $1, 3, 5$ are
dominated by the \ncl labels of vertices $10, 50, 30$ respectively.  The
following theorem states that the \ncl of a pattern vertex $u$ is dominated by
the \ncl of any of its representative vertex $v \in  R(u)$.

\begin{thm} Given any pattern vertex $u$, a representative vertex $v \in R(u)$
    and cost threshold $\alpha$, $\nclab{k}{u} \preceq \nclab{k}{v}$ for all $k
    \geq 0$.  \begin{myproof} Let $\phi$ be any isomorphism such that $\phi(u) =
        v$.  We prove the theorem by using induction on $k$.\\ \textbf{Base
        case:} $\nclab{0}{u} \preceq \nclab{0}{v} \iff \matij{C}{L(u)}{L(v)}
        \leq \alpha$ is true because $v \in R(u)$. \\ \textbf{Inductive
        Hypothesis:} Assume that $\nclab{k}{u} \preceq \nclab{k}{v}$ holds true
        for all $u \in \pat$ and $v \in R(u)$. \\ Now consider $\nclab{k+1}{u} =
        (A, B)$  and $ \nclab{k+1}{v} = (A', B') $, from theorem \ref{thm:khop}
        we know that $\khopcost{k+1}{B}{B'} \leq \alpha$, for all $k \geq 0$.
        Let $u'$ be a neighbor of $u$ and let $v' \in \phi(u')$ , then from the
        inductive hypothesis $\nclab{k}{u'} \preceq \nclab{k}{v'}$.  Therefore,
        the injective function $\phi$ maps the elements $a \in A$ to $\phi(a)
        \in A'$.  The theorem follows from the definition of the NL label.
    \end{myproof} \label{thm:ncl} \end{thm}

Based on the above theorem, a vertex $v$ can be pruned from \CR if $\nclab{k}{u}
\not\preceq \nclab{k}{v}$ for some $k \geq 0$. In Fig~\ref{fig:ncexample} ,
consider the vertices $3 \in \pat$ , $50 \in \db$ and let $\alpha = 0.5$. The
\ncl labels, $\nclab{0}{3} \preceq \nclab{0}{50}$ as $\matij{C}{B}{B} = 0 \leq
\alpha$.  Similarly it is also true for the pairs $(2, 20)$, $(4, 40)$ etc. It
follows that $\nclab{1}{3} \preceq \nclab{1}{50}$ as the neighbors $2, 4, 6$ can
be mapped to $20, 40, 60$ respectively and  the minimum cost of the matching the
$1$-hop label is $0.4$ which is less than the $\alpha$ threshold. But
$\nclab{2}{3} \not\preceq \nclab{2}{50}$ because the \ncl label $\nclab{1}{6}$
is not dominated by the \ncl label of $20, 40$ or  $60$ in the previous
iteration. So, there is no mapping between the neighbors of vertices $3$ and
$50$ in the current iteration. Hence, the vertex $50$ can be pruned from the
candidate representative set of vertex $3$.  Note that using the \khop label in
the same example will not prune the vertex $50$ because the minimum cost of
matching the \khop labels is within $\alpha$ as shown in table
\ref{tab:khop350}. Therefore, \ncl label is more efficient compared to \khop
label as it subsumes the latter label.

\subsection{Candidate set verification} \label{sec:verification} The pruning
methods based on the \khop and the \ncl labels start with a \CR and prune some
of the candidate vertices based on the conditions described in theorems
\ref{thm:khop} and \ref{thm:ncl}.  The verification step reduces \CR to \RS by
retaining only those vertices for which there exists an isomorphism $\phi$ in
which $\phi(u) = v$.  It does this by checking if the pattern $P$ can be
embedded at $v$ such that total cost of label mismatch is at most $\alpha$.


Let $w_p = u_0,\ldots, u_m$ be a walk in the pattern that covers each edge of
the pattern atleast once starting from vertex $u$ i.e. $u_0 = u$. Finding a path
that covers each edge atleast once is a special case of the chinese postman
problem \cite{chinesepostman} where the distance betweeb pair of vertices is
one. The following three conditions are satisfied iff the vertex $v$ is a
representative of the vertex $u$. i) there exists a walk $w_d = v_0,\ldots, v_m$
such that $v_i \in R(u_i)$ and $v_{i+1} \in R(u_{i+1})$, $\forall (v_i, v_{i+1})
\in w_d$.  ii) $v_i = v_j$ implies that $u_i = u_j$. iii)
$\sum\matij{C}{L(u_i)}{L(v_i)} \leq \alpha$ where $u_i \in \vp$. These
conditions can be verified by following the definition of approximate subgraph
isomorphism. Using an example we will show how these conditions can be used to
check definitely whether $v \in R(u)$.

Consider checking whether the vertex $30$ is a valid representative of the vertex
$1$ in the pattern in the figure~\ref{subfig:ex_sub} and let $\alpha = 0.5$ . The
walk $ w_p = 1, 2, 4, 3, 1$ covers each edge of the pattern atleast once. A
walk $w_d$, in the database in the figure~\ref{subfig:ex_db}, 
that satisfies the above three conditions should start by mapping the
vertex $1$ to $30$. The total cost of matching the labels till now is $0.2$ and
the budget available for matching other vertices in $P$ is $0.5 - 0.2 = 0.3$.
We will cover the walk $w_p$ one edge at a time. For any $(u_i, u_{i+1}) \in
w_p$, if $u_i$ and $u_{i+1}$ are mapped, then we verify that there is an edge
between the database vertices $v_i$ and $v_{i+1}$ to which the pattern vertices
are mapped.  If however, $u_{i+1}$ is not mapped then we map it to some vertex
$v_{i+1} \in R'(u_{i+1})$ and subtract the cost of this mapping from the
remaining $\alpha$ cost.  For example, in the first step we can map $2$ to $20$.
The remaining cost is then $0.3 - 0.2 = 0.1$. In the next step $(2,3)$, the
vertex $3$ cannot be mapped to any vertex in the database without violating the
third condition.  In such a case, we back track one step and choose a different
mapping say $10$ for the vertex $2$ which is the last vertex that was mapped.
Proceeding this way, we can arrive at the mapping corresponding to $\phi_{1}$ as
in table~\ref{subfig:ex_occur}.  This isomorphism not only guarantees that $30
\in R(1)$, it also implies that the verification check between the pairs $(2,
10), (3, 60)$ and $(4, 40)$ can be avoided because of the approximate isomorphism
$\phi_1$ that was found.  The above procedure can extended to enumerate the
complete set of isomorphisms.



\subsection{Label costs and dominance checking} \label{sec:labelcheck} 
Candidate representative vertices are pruned by checking for dominance 
relation between the \ncl labels of pattern vertex and that of candidate
vertex in the database. Comparing the \ncl labels requires i) computing the
cost between \khop labels ii) matching the neighbors of pattern vertex 
with neighbors of the candidate vertex. First problem can be formulated
as a minimum cost maximum flow in a network and the second as maximum matching
in a bipartite graph.

\medskip{\textit{Computing \khop label cost}:} To compute the minimum matching
cost between the \khop labels $\khopl{k}{u}$ and $\khopl{k}{v}$, we compute the
maximum flow with minimum cost in a flow network $F$ defined as follows.  Each
edge in $F$ is associated with a maximum capacity and the cost for sending a
unit flow across that edge.  The network contains a vertex for each label $l_u =
L(u')$ where $v' \in h_k(u)$ and a vertex for each label $l_v = L(v')$ where $v'
\in h_k(v)$. There is a directed between between source vertex ($s$) and each
$l_u$ with a zero cost and a capacity equal to the multiplicity of the $l_u$
i.e., the number of vertices in $h_k(u)$ that have the label $l_u$. Similarly
there is a directed edge between $l_v$ and the sink node ($t$). In addition,
there is a directed edge from $l_u$ to $l_v$ with a infinite capacity and a cost
equal to $\matij{C}{l_u}{l_v}$. The cost between the \khop labels is equal to
the minimum cost for maximum flow if the maximum flow is equal to
$|\khopl{k}{u}|$ and $\infty$ otherwise.  \\ Figure~\ref{fig:Hflow} shows the
flow network required to compute the minimum cost of matching the \khop labels
$\khopl{2}{2} = 4,6 $ and $\khopl{2}{20} = 40, 50, 60$ as shown in Table
\ref{tab:khop220}. The labels of vertices in the \khop labels are $C,D$ and $B,
B, A$ respectively. There is an edge from $s$ to each of $C, D$ with zero cost
and maximum capacity of one.  Similarly, there is an edge from each of $A, B$ to
the sink vertex $t$ with zero cost and maximum capacity of one and two
respectively. There is an edge from $C, D$ to each of $A, B$ with cost equal to
the corresponding entry in the cost matrix $C$. The maximum flow in the network
is two and the minimum cost of sending two units of flow $0.4$ is achieved by
pushing a unit flow along the paths $s, C, B, t$ and $a, D, A, t$.  Therefore,
the cost of matching the labels $\khopl{2}{2}$ and $\khopl{2}{20}$ is $0.4$. It
implies that the vertex $4$ with label $C$ can be matched to either $40$ or $50$
and the vertex $6$ to $60$.

\medskip{\textit{Dominance check}: } Consider the 
\ncl labels $\nclab{k+1}{u} = (A, B)$ and  $\nclab{k+1}{v} = (A', B')$, the cost matching
$B$ and $B'$ can be computed using the above the network formulation.
Finding an injective function $f\!\!:A \rightarrow A'$ such that $a \preceq
f(a)$ , is equivalent to find a matching of size $|N(u)|$ in the bipartite graph
with edges $(a, a')$, for all $a \in A$ and $a \preceq a'$.
The \ncl label $\nclab{k}{v}$ therefore dominates $\nclab{k}{u}$ if the
cost between the \khop labels is within $\alpha$ and the size of maximum
bipartite matching is $|N(u)|$.

\medskip{\textit{Optimization}: } The candidate pattern may contain groups of
symmetric vertices that are indistinguishable with respect to the \khop label.
In such a scenario, the candidate representative sets of all these vertices are
exactly the same. Utilizing the symmetry, we apply the pruning label strategy only on
one vertex per symmetry group and replicate the results for all other vertices
in the group. For example, the vertices $10$ and $40$ in
figure~\ref{subfig:ex_sub} are symmetric and the representative sets $R'(10)$
and $R'(40)$ are exactly the same.  In abstract algebra terms such  groups are
called orbits of the graph and can be computed using nauty algorithm
\cite{nauty}. 
Even
though computing the orbits is expensive, we can avoid $ (|g|-1) \times |\CR|$
\ncl label cost computations where $g$ is the size of an orbit. Note that
we find the orbits only for the pattern which is usually very small compared
to the database graph.
%Note that the payoff is zero if all the vertex orbits are of size $1$.

\subsection{Precomputing database \khop labels} The \khop label of the database
vertices is independent of the candidate pattern.
Also, the flow network to compute the cost of matching the \khop labels requires
only the aggregate information about the number of vertices of a given label.
Hence, we can precompute the \khop label of the database vertices and store them
in the memory. The following theorem proves that computing \khop label is
expensive.

\begin{thm} k-reachable (KR) : Given a graph $G$, $k$ and $u \in \vg$. Compute
    $\khopl{k}{u}$.  KR cannot be solved in polynomial time unless $P =
    NP$.

\begin{myproof} We prove this by reducing hamiltonian path (HP) to KR.
    Hamiltonian Path : Given a graph $G$, is there a simple path of length
    $|\vg|-1$ i.e. is there a path that visits each and every vertex exactly
    once. The problem of finding a Hamiltonian path is proven to be an
    NP-Complete problem \cite{npcomplete}.\\ Assume that algorithm $X(k)$ can
    compute KR in polynomial time. Let $|\vg| = n$ and $u$ be the starting
    vertex in HP if it exists.  Given an instance of HP, we first get a vertex
    $v$, $\kpath{u}{v}{n-1}$ using $X(n-1)$. The vertex $v$ is removed from the
    graph and we find a vertex $v'$ such that $\kpath{u}{v'}{n-2}$ and $(v', v)
    \in \eg$. We repeat this process $n-1$ times. If at any stage $X(j) = \{\}$
    then we restart from a different starting vertex. The vertices selected in
    each iteration lie on a path of length $n-1$ if it exists. If there is
    polynomial time algorithm for KR then HP could be solved in polynomial time
    by reducing it to KR. Hence, KR cannot be solved in polynomial time unless
    P = NP.
\end{myproof}
\end{thm}

To compute \khop label of a vertex $u$, we check for each vertex $v$ whether $v
\in \kpath{u}{v}{k}$ by enumerating all possible $k$ length paths until a path
is found.  This procedure is exponential, we therefore fix a maximum value
$k_{max}$ and use the \khop label based pruning only for values of $k \leq
k_{max}$.  It only takes a couple of minutes to compute the \khop label for $k
\leq 6$ for all the vertices in the database graph. This is significantly less
than the overall run time of the algorithm. Once $\khopl{k}{u}$ is computed we
store in memory only the tuples $(m, l)$ where $m$ is the number of vertices $v
\in \khopl{k}{u}$, $L(v) = l$. The total amount of main memory required to store
the precomputed \khop labels is O($|\vg| \times |\Sigma| \times k_max$).

\subsection{Complexity} Memory and time complexity of the entire algorithm.



\begin{figure}[!h]
    \centering
\scalebox{0.6}[0.6]{
  \psset{unit=0.85in}
  \newcommand\arc[4]{\ncline{#1}{#2}{#3}\ncput{\colorbox{gray!40}{#4}}}
      \begin{pspicture}(0,1)(5,3)
        \cnodeput[doubleline=true](1,2){src}{s}
        \cnodeput(2,1){n1}{C}
        \cnodeput(2,3){n2}{D}
        \cnodeput[doubleline=true](5,2){sink}{t}
        \cnodeput(4,1){n4}{B}
        \cnodeput(4,3){n5}{A}
        \arc{->}{src}{n1}{$1,0$}
        \arc{->}{src}{n2}{$1,0$}
        %\arc{->}{n1}{n4}{$1$}
        \ncline{->}{n1}{n4}\ncput[npos=0.5]{\colorbox{gray!40}{$\infty,0.1$}}
        \ncline{->}{n1}{n5}\ncput[npos=0.3]{\colorbox{gray!40}{$\infty,1$}}
        \ncline{->}{n2}{n4}\ncput[npos=0.3]{\colorbox{gray!40}{$\infty,0.6$}}
        \ncline{->}{n2}{n5}\ncput[npos=0.5]{\colorbox{gray!40}{$\infty.0.3$}}
        \arc{->}{n4}{sink}{$2,0$}
        \arc{->}{n5}{sink}{$1,0$}
        %\arc{->}{n6}{sink}{$1$}
      \end{pspicture}
    }
    \caption{Flow network for \khopl{2}{2} and \khopl{2}{20}}
	\label{fig:Hflow}
\end{figure}
