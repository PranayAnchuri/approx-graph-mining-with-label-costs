\section{Introduction}
Graphs are a natural way to model many of the modern complex
datasets that typically have interlinked entities connected with various
relationships. Examples include different types of networks, such as
social, biological and technological networks. Tools for rapidly
querying and mining graph data are therefore in high demand. Our focus
is on graph pattern discovery methods that can simultaneously consider
both the structure and content (e.g., node labels).

Whereas frequent graph mining has long been a well studied problem, most
of the prior work has focused on exact pattern discovery.
Graph mining involves two main steps. The first step is to generate
non-duplicate candidate patterns, and the second is to compute the
frequency of each candidate pattern. The former task requires graph
isomorphism testing, whereas the latter requires subgraph isomorphism
checking, since we need to count all the occurrences of a smaller graph
within a much larger graph (or a set of graphs). Many efficient methods
have been proposed for mining exact labeled graph patterns, including
both complete search and sampling based
approaches~\cite{gSpan,HWP03,kuramochi2005ffp,FSG01,IWM03,2009-graphsampling}.
These exact methods require that there be an exact match between the
labels of nodes in the candidate pattern and in the database graph. This
can potentially miss many patterns where nodes may share a high label
similarity, but may not match exactly. This is specially true for more
complex labels (e.g., text data), or in cases where the nodes represent
some real-world objects (e.g., proteins, IT infrastructure nodes), where
it may be possible to easily design a meaningful cost or distance matrix  between node ``labels''. Unfortunately, exact isomorphism
based methods cannot leverage the rich information from the cost matrix.
What is required is a new class of algorithms that can mine
frequent approximate patterns via approximate subgraph isomorphism that
satisfies some bound on the overall cost of the match between a candidate
and the database graph(s). Only a few methods have tackled this
problem~\cite{gapprox,JiaZH11,RAM2008}, but they typically enumerate all
isomorphisms, and are therefore not scalable to large graphs due to the
combinatorial explosion in the number of isomorphisms.

In this paper we present a new approach to mine frequent approximate
patterns in the presence of a cost matrix between the labels. In
particular we make the following contributions:
\begin{itemize}
\item We propose a novel approach to effectively prune the space of
  approximate labeled isomorphisms. Instead of enumerating all the
  possible isomorphisms, we maintain a set of representatives (nodes in
  the database that match a candidate pattern) that is linear in the
  database and pattern size. Pruning is applied on this set to narrow
  down the search to only viable mappings.
\item We propose several iterative label updating methods that yield
  derived cost matrices on the basis of which more effective 
  pruning can be achieved. These are based on $k$-hop labels, neighbor
  concatenated labels and a combination of the two.
\item Our method handles both arbitrary as well as binary cost matrices.
\item We place our work within the pattern sampling paradigm, 
  thereby avoiding complete search, which can be practically 
  infeasible in real-world
  graphs, not to mention the information overload problem.
\end{itemize}
We study the effectiveness of the proposed methods on three real-world
datasets. The first is a configuration management database graph, 
where the nodes represents
entities comprising the IT infrastructure and the link represents
relationships between them; approximate mining yields a richer de-facto IT
policies in the company. The second dataset is a graph dataset
representing 3D protein structures; mined patterns represent approximate
motifs. The last dataset comes from a protein interaction network, where
the nodes are proteins and edges indicate whether they interact
physically (i.e., they may bind together or they may be part of the
same protein complex); the mined approximate patterns represent
molecular subnetworks and molecular machines (the protein complexes)
that take part in important cellular processes. We show that our
proposed techniques are indeed scalable and fruitful, allowing us
to mine interesting approximate graph patterns from 
large real-world graphs.
