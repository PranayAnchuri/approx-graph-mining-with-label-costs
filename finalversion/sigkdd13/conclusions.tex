% conclusions and future work
\section{Discussions and Conclusions}
\label{sec:conclusions}

We presented an effective approach to mine approximate frequent subgraph
patterns from a single large graph database in the presence of
a label cost matrix.

There are two main parameters in our method: $K$, the number of random
walks, and $\alpha$ the cost threshold. The value of $K$ is directly
proportional to the number of maximal approximate patterns we desire,
and is relatively easy to set.
On the other hand, choosing an appropriate value of
$\alpha$ is very important as it affects the quality of
patterns mined. Depending on the application domain
and the purpose of the graph mining, let $t$ be the number of
vertices in the pattern for which we allow label mismatches
in the subgraph isomorphism. One reasonable value of
$\alpha$ is $t \times IMQ$ where $IMQ$ is the inter-quartile mean
i.e., the mean of the entries between the first quartile ( $25^{th}$
percentile) and the third quartile ( $75^{th}$ percentile)
of the entries in the cost matrix arranged in sorted order. 
$t$ can be chosen by first
enumerating maximal patterns with $\alpha = 0$ and computing the 
average size $m$ of the maximal patterns mined from the graph.
The value of $t$ then is a fraction of the average size $m$.
Care has to be taken not to choose a very large $\alpha$ as it leads to
patterns of poor quality and also increases the run time of the
algorithm significantly as can be seen in Table \ref{tab:scop_alpha}.

In terms of future work, we plan to increase the efficiency of our
method by exploiting parallelism. Obviously different walks can be
carried out in parallel. However, more interesting is the
parallelization of the approximate isomorphism generation and
label-based pruning steps, including verification. We also
want to explore the idea of label based pruning for more
general definitions of approximate isomorphism including 
edge mismatches.

