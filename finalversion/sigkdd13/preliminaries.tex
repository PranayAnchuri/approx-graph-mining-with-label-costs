\section{Preliminaries}

An undirected labeled graph \db is represented as a tuple $ \db =
(\vg,\eg,L) $ where $\vg$ is the set of vertices, $\eg$ is the set of
edges and $L\!\!: \vg \rightarrow \Sigma $ is a function that maps
vertices to their labels.  The neighbors of a vertex $v$ are given as $
N(v) = \{ u | (u,v) \in \eg \} $.  A
{\em path} of length $k$ in a graph $\db$ is a sequence of vertices $v_0,\ldots,v_k$
such that $v_i \in \vg$, $(v_i,v_{i+1}) \in \eg$ and $v_i \neq v_j$ for
$i \neq j$.
We use $\kpath{u}{v}{k}$ if there is a $k$ edge path
between $u$ and $v$.
%It covers the vertices $\cup_{i=0}^{k} \{v_i\}$ and the edges $\cup_{i=0}^{k-1}
%\{(v_i,v_{i+1})\}$.  
A path is called a walk if the vertices
are allowed to repeat.

\smallskip\noindent{\bf Cost matrix:}
We assume that there is a cost matrix 
$\costmat{C} \!\!:\Sigma^{2} \rightarrow \mathbb{R}_{\geq 0} $. 
The entry $\labcost{C}{l_i}{l_j}$
denotes the cost of matching the labels $l_i$ and $l_j$. Although it is 
not required by the algorithm, $\mathcal{C}$ is usually symmetric and the diagonal
entries are $0$

\smallskip\noindent{\bf Approximate subgraph isomorphism:}
A graph $S = (V_S,E_S,L)$ is a subgraph of \db, denoted $S \subseteq
\db$, iff $V_S \subseteq \vg$, and $E_S \subseteq \eg$.  Given a database
graph $G$ and a pattern $P = (V_P,E_P,L)$, a function $\phi\!\!: V_P \to
V_G$ is called an {\em unlabeled subgraph isomorphism} provided $\phi$
is an injective (or one-to-one) mapping such that $\forall (u,v) \in
E_P$, we have $(\phi(u),\phi(v)) \in \eg$. That is, $\phi$ preserves the
topology of $P$ in $G$. Define the cost of the isomorphism as follows:
$\isocost{C}{\phi} = \sum_{u \in \vp} \labcost{C}{L(u)}{L(\phi(u))}$, that is, the
sum of the costs of matching the node labels in $P$ to the corresponding
node labels in $G$.  We say that $\phi$ is an {\em approximate subgraph
isomorphism} from $P$ to $G$ provided its cost $C(\phi) \le \alpha$,
where $\alpha$ is a user-specified threshold on the total cost. In this
case we also call $P$ an approximate pattern in $G$. Note
that if $\alpha = 0$, then $\phi$ is an unlabeled subgraph isomorphism
between $P$ and $G$. From now on, \textit{isomorphism}
refers to \textit{approximate subgraph isomorphism} unless specified
otherwise.


%%% Example graphs to illustrate the notation used in the paper
\begin{figure}[!h]
\vspace{-0.25in}
\captionsetup[subfloat]{captionskip=15pt}
  \subfloat[Database Graph $\db$] {
    \label{subfig:ex_db}
  \scalebox{0.9}{
    \begin{pspicture}(-1,0)(3,3)
      \putNode{1}{2}{n1}{A}{90}{10}
      \putNode{0}{1}{n2}{A}{180}{20}
      \putNode{1}{1}{n3}{B}{225}{30}
      \putNode{2}{1}{n4}{A}{0}{40}
      \putNode{0}{0}{n5}{C}{270}{50}
      \putNode{1.5}{0}{n6}{C}{270}{60}

      \ncline{-}{n1}{n2}
      \ncline{-}{n1}{n3}
      \ncline{-}{n1}{n4}
      \ncline{-}{n2}{n3}
      \ncline{-}{n3}{n4}
      \ncline{-}{n3}{n6}
      \ncline{-}{n4}{n6}
      \ncline{-}{n2}{n5}
    \end{pspicture}
	}}
  % Graph subgraph isomorphic to the database
  \subfloat[Pattern $P$] {
    \label{subfig:ex_sub}
  \scalebox{0.9}{
    \begin{pspicture}(0.5,1)(3,3)
      \putNode{2}{3}{n1}{A}{90}{1}
      \putNode{1}{2}{n2}{B}{180}{2}
      \putNode{3}{2}{n3}{C}{0}{3}
      \putNode{2}{1}{n4}{A}{0}{4}
      \ncline{-}{n1}{n2}
      \ncline{-}{n1}{n3}
      \ncline{-}{n2}{n4}
      \ncline{-}{n3}{n4}
    \end{pspicture}
	}}
  \newline
\captionsetup[subfloat]{captionskip=5pt}
  \subfloat[Cost Matrix] {
    \label{subfig:ex_match}
    \begin{tabular}{|c|c|c|c|c|}
      \hline
      \costmat{C} & A &  B & C & D \\
      \hline
      A & $0$ & $0.2$ & $0.6$ & $0.1$ \\
      \hline
      B & $0.2$ & $0$ & $0.4$ & $0.5$ \\
      \hline
      C & $0.6$ & $0.4$ & $0$ & $0.2$ \\
      \hline
      D & $0.1$ & $0.5$ & $0.2$ & $0$ \\
      \hline
    \end{tabular}
  }
  % Approximately isomorphic
  \subfloat[Approximate Embeddings] {
      \label{subfig:ex_occur}
      \begin{tabular}{|l|c|c|c|c|c|}
		\hline
		 & \multicolumn{5}{|c|}{$\phi$}\\ 
        \hline
		$P$ & $1$ & $2$ & $3$ & $4$ & cost \\
        \hline
        $\phi_1$ & $30$ & $10$ & $60$ & $40$ & $0.4$ \\
        $\phi_2$ & $40$ & $10$ & $60$ & $30$ & $0.4$\\
        \hline
      \end{tabular}
  }
  % occurrences of the approx pattern
  \caption{ \protect\subref{subfig:ex_db}: sample database graph $G$, 
    \protect\subref{subfig:ex_sub}: approximate pattern $P$.
	\protect\subref{subfig:ex_match}: cost matrix.
	\protect\subref{subfig:ex_occur}: approximate embeddings 
	of $P$ in $G$.
  }
  \label{fig:ex1}
\end{figure}

\smallskip\noindent{\bf Representative set :}
Given a node $u \in \vp$, its {\em representative set} in the database
graph $G$ is the set 
$$R(u) = \{ v \in \vg |\; \exists \phi, \text{ such
that } C(\phi) \le \alpha \text{ and } \phi(u) = v \}$$ 
That is, the representative set of $u$ comprises all nodes in $G$ that
$u$ is mapped to in some approximate isomorphism.  
Figure
\ref{fig:ex1} shows an example database, a cost matrix, an approximate
pattern, and its approximate subgraph isomorphism for $\alpha=0.5$.
There are only two possible approximate isomorphisms from $P$ to $G$, as
specified by $\phi_1$ and $\phi_2$. For example, for $\phi_1$, we have
$\phi_1(1) \to 30$, $\phi_1(2) \to 10$, $\phi_1(3) \to 60$, and
$\phi_1(4) \to 40$, as seen in Table~\ref{subfig:ex_occur}. 
The cost of the isomorphism is 
$\mathcal{C}(\phi_1) = 0.4$, since 
$\mathcal{C}(L(1),L(30)) + \mathcal{C}(L(2),L(10)) + \mathcal{C}(L(3),L(60)) + \mathcal{C}(L(4),L(40)) 
= \mathcal{C}(A,B) + \mathcal{C}(B,A) + \mathcal{C}(C,C)+ \mathcal{C}(A,A) = 0.2+0.2+0+0 = 0.4$. 
The representative
set for node $1 \in \vp$ is $R(1) = \{30, 40\}$. 

%%%% Define a frequent pattern

\smallskip\noindent{\bf Outline of our Approach:} The two main steps in
approximate graph mining are candidate generation and support
computation. With candidate generation the search space of the frequent
patterns is explored. For each candidate that is generated, we can check
whether it is frequent by computing its support.

Given a pattern with $k$ vertices, the maximum number of possible
isomorphisms are $k! \times {|\vg| \choose k} $.  It is therefore
infeasible to either enumerate or store the complete set of
isomorphisms. Computing and storing the representative sets is a
compromise that will enable us to decide efficiently if a candidate is
frequent.

%
%The rest of the paper is structured as follows. In section
%\ref{sec:representative} we propose methods for computing the representative
%sets. In section, \ref{sec:mining} we describe how the idea of representative
%sets can be combined with different candidate generation and support computation
%mechanisms to yield mining algorithms with different properties. In section
%\ref{sec:results}, we present detailed experimental results to show that the
%proposed methods can be used to mine interesting and useful frequent patterns
%from three real world datasets.  We discuss the related work in section
%\ref{sec:relatedwork} and conclude in section \ref{sec:conclusions}
%

%%%%%%%%% the definition of support is moved to a later section %%%%%%
\if0
Define the {\em
support} of pattern $P$ in a database graph $G$ as 
$$sup(P) = \min_{u \in
\vp} \{ |R(u)| \}$$
That is, the minimum cardinality over all
representative set of nodes in $P$.  A pattern $P$ is called {\em
frequent} if $sup(P) \geq minsup$, where $minsup$ is a user defined
support threshold.  $P$ is {\em maximal} iff $P$ is frequent and there
does not exist any supergraph of $P$ that is frequent in $G$.  

\smallskip
\noindent{\bf Problem Statement:} Given a database graph \db, minimum
support $minsup$, maximum allowed cost $\alpha$, and an integer $k$, our
goal is to mine $k$ maximal frequent approximate patterns.  Figure
\ref{fig:ex1} shows an example database, a cost matrix, an approximate
pattern, and its isomorphisms for $\alpha=0.5$.
There are only two possible isomorphisms from $P$ to $G$, as
specified by $\phi_1$ and $\phi_2$. For example, for $\phi_1$, we have
$\phi_1(10) \to 3$, $\phi_1(20) \to 1$, $\phi_1(30) \to 6$, and
$\phi_1(40) \to 4$, as seen in Table~\ref{subfig:ex_occur}. 
The cost of the isomorphism is 
$C(\phi_1) = 0.4$, since 
$C(L(10),L(3)) + C(L(20),L(1)) + C(L(30),L(6)) + C(L(40),L(4)) 
= C(A,B) + C(B,A) + C(C,C)+ C(A,A) = 0.2+0.2+0+0 = 0.4$. 
The representative
set for node $10 \in \vp$ is $R(10) = \{3, 4\}$. However, the
support of $P$ is $sup(P) = 1$, since node $20 \in \vp$ has only one
mapping in $G$, namely $R(20) = \{1\}$.
\fi
